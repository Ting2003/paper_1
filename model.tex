\section{Component Model for Transient analysis}
In power grid model for transient, there are two new components: capacitance and inductance. There are three equivalent linear models for
these two devices: Forward Euler (FE), Backward Euler (BE), and Trapezoidal model (TR). The original matrix for DC analysis is symmetrical potitive definite (SPD), thus cholesky decomposition method can be utilized to do the factorization. Cholesky decomposition is not only
on average 2 times faster than LU decompoisition method, but also saves about half the memory for storing the matrix. As a result, it is
highly preferrable to make the conductance SPD, so that cholesky decomoposition method can be utilized. In transient 
analysis, components as capacitance and inducatance are considered in conductance matrix. Whether the conductance matrix is SPD depends on
the models used for these two components. 
\begin{figure*}[htbp]
  \centering
  \includegraphics[width = 0.8\textwidth]{TR_model_new.pdf}
  \caption{Companion models for capacitor and inductor.}
  \label{TR_model}
\end{figure*}

Figure~\ref{TR_model} shows the two components and their three models. There are two representations of BE and TR: Norton and Trevin form, as shown 
in Figure~\ref{TR_model}. In FE model or Thevenion form of BE and TR models, an equivalent voltage source is introduced for at least one component. 
With any of these models, the conductance matrix is not SPD. This can be proved as follows: 

Suppose there is a voltage source between node $i$ and node $j$. The matrix before stamping the voltage source has dimension of $nxn$. 
After stamping this voltage source, the new matrix is shown as follows:
\begin{figure}[htbp]
 \centering
 \includegraphics[width=0.3\textwidth]{SPD_matrix_new.pdf}
 \caption{Conductance matrix after stamping companion model}
 \label{SPD_matrix}
\end{figure} 
%use equations to prove that the BE or FE modeled matrix is not SPD%

On the other hand, using the Norton form of TR or BE companion models for both capacitors and inductances will not introduce any 
voltage source. The introduced components are resistors and currents, which will not effect the SPD characteristic of conductance matrix. 
 
Besides, FE and BE model will change the effective number of nodes in power grid. As a result, the matrix sizes for DC and transient analysis are different. All the nodes have new indexes, and not any information of old matrix can be reused to form the new matrix. In fact, as 
the amount of capacitance and inductance only occupys a small ratio of the grid, and most of the nets
information forming the matrix are identical between DC analysis and transient analysis, it is highly preferable to get the new matrix by
only modifying capacitance and inductance nets. By using Norton form of TR or BE companion model of inductance and capacitance, no extra 
node 
will be introduced into the grid. All the nodes keeps their indices unchanged. As a result, the new matrix can be gained by only stamping
the capacitance and inductance equivalent components and doing a little modification to the old items of the original matrix. With this
model, the setting up work is much more easier and time saving, which is especially true for large size of power grids. Considering that 
Trapezoidal integration is in general more accurate than either FE or BE\cite{ECE552}, TR model is considered in this paper.

Based on the above consideration, TR companion models of capacitors and inductances in Norton form are utilized in this paper. The 
resulting matrix is SPD, and will be decomposed with cholesky method.