\section{Methodology}
  \subsection{Sparse Vector}
   %first show the example and cite%
    There are two stages in solving a linear system: matrix decompisition and forward (FE) and backward (BE) substitution stages. In matrix 
decomposition,
, the sparsity of matrix can greatly help reducing computation intensity. In the same way, the sparsity of vectors can also help reducing
the computation intensity. \cite{Tinney} illustrates the theory of sparse vector methods. Suppose we have a linear system with 20 nodes.
The conductance matrix has already been decomposed, which is 
\begin{equation}
	A=LL'\label{eq3.1}
\end{equation}
Forward and backward substitution are required to obtain the final solution. The equations for these two operations are as follows:
\begin{align}
	LY &=b  &(FE)\label{eq3.2}\\
	L'X &= Y&(BE)\label{eq3.3}
\end{align}

  %figure 1: elimination tree of A%
Figure 1.(a) Shows the $L$ of the decomposed matrix $A$ and the corresponding elimination tree. Assuming that only $y_{12}, y_{14}$ is 
non-zero, and only $x_{10},x_{17}$ is required for finding solution. Then, the items in $Y$ and $X$ that need to be computed can be 
presented along the following path, which represents their order of computation.
\begin{align}
&\{12, 15, 14, 17, 18, 19, 20\} &(FE)\label{eq3.4}\\
&\{20, 19, 18, 17, 13, 10\}&(BE)\label{eq3.5}
\end{align}
In this paper, we define the above two paths as ``forward path" and ``backward path". By utilizing the sparse characteristic of vectors,
only 6 or 7 items are needed to compute instead of 20. It can be seen that by utilizing the sparcity characteristic of vectors, a lot of
savings in FE and BE can be obtained. The amount of savings of computation depends on the number of nodes in the paths, which depends on 
the matrix structure. 
 
In transient analysis, it can be seen that as long as the right parameter values of capacitor and inductance are gained, it is 
sufficient to 
perform the computation of next time step. As a result, the voltage values of neighboring nodes of capacitor and inductance should be 
known. Besides, there are also tens of observation nodes, whose voltage values need to be known at each time step. The voltage values of
these two part of nodes are all information we need to get at each time step. Considering that the
total ratio that the above two parts of nodes occupyies is only a small amount of the total number of nodes in power grid, the equivalent
vector $X$ is very sparse. On the other hand, $b$ depends on the number of current sources in the grid, which is very small a ratio 
compared to the total number of nodes. $b$ is very sparse as well. With this sparse characteristic in vector $b$ and $X$, we can build the
forward and backward paths and save part of the effort in computation. This effort is worthwhile for transient analysis, because by 
reusing the paths, saving in computation is accumulated as time step goes by.

The challenging work is to build the forward and backward paths. \cite{Tinney} provides a path building method, where the first path is 
obtained on the first step. Then, the next path segments will be inserted into the original path as a whole unit. The inserting point is 
before the cross point of new path segment and original one. The resulting path is shown in equation \eqref{eq3.4} and \eqref{eq3.5}. It
can be seen that the resulting index of rows/columns in the path is not mononically ascending or descending. This will cause trouble in
later substitution stage, because it is not possible to finish traversing the path with a single for loop. In order to reduce the 
complexity, the indexes in the path have to be mononically ascending in FF and mononically descending in BF. Another more serious problem
is that this method can not detect the case where a node does not have any decedent in the elimination tree, such as column $5$ in figure 
1. Because diagonal element does not have any decedent element, no insertion point can be find with the above method. As a result, the 
path is showing incomplete information, and will lead to wrong solution. 

In order to overcome the above limitations, a new method is proposed in this paper. It is illustrated in Figure 2.
%Figure 2: showing the algorithms of path building method%
After building the first path, the next segment start from the node with minimum index and will stop under two cases: 
\begin{enumerate}[1)]
\item Current node has a descedent and the descedent node is already in the formed path; 
\item Current node has no descedent.
\end{enumerate} 
By considering the second case, the node that has no descedent is included to form the path. We build a list to record all the nodes in 
the segments. To make the later substitution stage being finished in a for loop, we sorted the nodes in the list. The sorted list are then
inserted into the original path obtained from first step. Since the nodes in the list are in asceding order, so does the original path, 
the searching and insertion will start from the position that the last node is inserted, instead of the head of the path. The cost for
insertion is $O(n)$, where $n$ is the number of nodes in the matrix. Notice that we will only sort an array with length of the list, 
instead of all the nodes in the matrix. If the first path is very long, we only need to sort a short array, which reduces a lot of effort.
%In the experiment result, need a table to show the ratio saved with this technique, and the speed for building the paths% 

 \subsection{Solution Mapping}
In order to reduce fill-ins, reordering is usually taken in matrix decomposition process. The ordering in decomposed 
matrix and the original one is different. In order to get the right result, corresponding reordering must be performed to the right hand 
side vector. Regular solving process is as follows:
\begin{enumerate}[1)]
\item Reordering the right hand side vector in the same way of decomposing matrix.
\item Perform forward and backward substitution to get the reordered solution.
\item Reordering the solution vector back in the inverse way as that of the right hand side.
\end{enumerate}
As we can see, in each time step, 2 reordering operations required to get the right solution. If we utilize the regular solving process in each
time step, for transient analysis of large power grid, the reordering process may cost a lot of time. It is highly perferable to 
eliminate the reordering operation. It is possible to realize this idea case: after factorizing the conductance matrix, the mapping 
relationship between the original and new order of each node is recorded. For each node on power grid, the original index is renewed
into the new one. Then, the right hand side is restamped based on the new ordering. Through these operations, the right hand side vector is
in the same ordering with the decomposed matrix. The equivalent linear system for solving stage is as follows:
\begin{equation}
A'x = b'
\end{equation}
where $A'=LL'$ is the decomposed matrix without reordering, and $b'$ is the right hand side without reordering.

As a result, the 2 reordering operations are not needed. Only the FE and BE operations are executed for all the time steps.
%In Experiment result, need a table to show the time savings with the reordering technique% 
  \subsection{Memory Supernode Solving}
%find and cite some papers about supernode technique, a brief introduction of supernode solving%
%need to talk about regular supernode solving process: judgement is required to be performed in each time step%
In this paper, a modified supernode technique named ``Memory Supernode Solving" is proposed for efficient transient simulation. After 
factorizing the conductance matrix, the number of columns contained in each supernode is gained by judging the characteristic among 
neighboring columns. This information is stored in an array. Then, in each time step, corresponding functions for different type of 
supernode are directly called by getting data from the stored array. No complex judgement is required. This is effective for transient
analysis, in which multi step computation is required.  
%In experiment result, need a table to show the time savings with Memorized supernode solving technique%
  \subsection{Optimized Current Restamping}
In transient analysis, current varies among different time steps. Because of this change, vector $b$ in equation $AX=b$ needs to be 
modified at different time steps. It should be noticed that part of the currents stay static for a period. We do not need to modify
the items that correspond to these current sources. Instead, in each time step, we will locate those current sources whose value are
different from last step, and only modify the corresponding items in the right hand side. 
  \subsection{Multiple threads}
There are usually several sub circuits in power grids: VDD, GND and so on. These sub circuits are totally disjoint and can be solved
in parallel. In this paper, multiple threads are used to parallelize the simulation. One thread handles one sub circuit. The number of
speed up is almost identical to the number of sub circuits in the power grids.
%talking about dynamic schedule, and why further speed up can not gained with this stra